\documentclass[11pt]{charter}

% El títulos de la memoria, se usa en la carátula y se puede usar el cualquier lugar del documento con el comando \ttitle
\titulo{Sistema de monitoreo remoto de estaciones de telefonía celular} 

% Nombre del posgrado, se usa en la carátula y se puede usar el cualquier lugar del documento con el comando \degreename
%\posgrado{Carrera de Especialización en Sistemas Embebidos} 
\posgrado{Carrera de Especialización en Internet de las Cosas} 
%\posgrado{Carrera de Especialización en Intelegencia Artificial}
%\posgrado{Maestría en Sistemas Embebidos} 
%\posgrado{Maestría en Internet de las cosas}

% Tu nombre, se puede usar el cualquier lugar del documento con el comando \authorname
\autor{José Ramón Castiñeiras} 

% El nombre del director y co-director, se puede usar el cualquier lugar del documento con el comando \supname y \cosupname y \pertesupname y \pertecosupname
\director{Esp. Ing. Emilio Moretti}
\pertenenciaDirector{FIUBA} 
% FIXME:NO IMPLEMENTADO EL CODIRECTOR ni su pertenencia
%\codirector{} % si queda vacio no se deberíá incluir 
%\pertenenciaCoDirector{}

% Nombre del cliente, quien va a aprobar los resultados del proyecto, se puede usar con el comando \clientename y \empclientename
\cliente{Ing. Luis Parra}
\empresaCliente{Tower One Wireless}

% Nombre y pertenencia de los jurados, se pueden usar el cualquier lugar del documento con el comando \jurunoname, \jurdosname y \jurtresname y \perteunoname, \pertedosname y \pertetresname.
\juradoUno{Nombre y Apellido (1)}
\pertenenciaJurUno{pertenencia (1)} 
\juradoDos{Nombre y Apellido (2)}
\pertenenciaJurDos{pertenencia (2)}
\juradoTres{Nombre y Apellido (3)}
\pertenenciaJurTres{pertenencia (3)}
 
\fechaINICIO{25 de agosto de 2020}		%Fecha de inicio de la cursada de GdP \fechaInicioName
\fechaFINALPlanificacion{24 de Agosto de 2021} 	%Fecha de final de cursada de GdP
\fechaFINALTrabajo{22 de setiembre de 2021}		%Fecha de defensa pública del trabajo final


\begin{document}

\maketitle
\thispagestyle{empty}
\pagebreak


\thispagestyle{empty}
{\setlength{\parskip}{0pt}
\tableofcontents{}
}
\pagebreak


\section{Registros de cambios}
\label{sec:registro}


\begin{table}[ht]
\label{tab:registro}
\centering
\begin{tabularx}{\linewidth}{@{}|c|X|c|@{}}
\hline
\rowcolor[HTML]{C0C0C0} 
Revisión & \multicolumn{1}{c|}{\cellcolor[HTML]{C0C0C0}Detalles de los cambios realizados} & Fecha      \\ \hline
1.0      & Creación del documento                                          & 27/08/2020 \\ \hline
1.1      & Carga inicial                                                   & 03/09/2020 \\ \hline
1.2      & Se completa hasta punto 6									   & 06/09/2020 \\ \hline
1.3      & Se agrega Dir. de Proyecto e Historias de usuario iniciales	   & 12/09/2020 \\ \hline
1.4      & Se realizan correcciones sugeridas por docente P.Bos			   & 13/09/2020 \\ \hline
%		   Con texto partido \newline
%		   En varias líneas \newline
%		   A propósito                                                     & dd/mm/aaaa \\ \hline
\end{tabularx}
\end{table}

\pagebreak



\section{Acta de constitución del proyecto}
\label{sec:acta}

\begin{flushright}
Buenos Aires, \fechaInicioName
\end{flushright}

\vspace{2cm}

Por medio de la presente se acuerda con el Ing. \authorname\hspace{1px} que su Trabajo Final de la \degreename\hspace{1px} se titulará ``\ttitle'', consistirá esencialmente en el prototipo preliminar del sistema mencionado, y tendrá un presupuesto preliminar estimado de 612 hs de trabajo y \textcolor{red}{\$2000} dólares americanos; con fecha de inicio \fechaInicioName\hspace{1px} y fecha de presentación pública \fechaFinalName.

Se adjunta a este acta la planificación inicial.

\vfill

% Esta parte se construye sola con la información que hayan cargado en el preámbulo del documento y no debe modificarla
\begin{table}[ht]
\centering
\begin{tabular}{ccc}
\begin{tabular}[c]{@{}c@{}}Ariel Lutenberg \\ Director posgrado FIUBA\end{tabular} & \hspace{2cm} & \begin{tabular}[c]{@{}c@{}}\clientename \\ \empclientename \end{tabular} \vspace{2.5cm} \\ 
\multicolumn{3}{c}{\begin{tabular}[c]{@{}c@{}} \supname \\ Director del Trabajo Final\end{tabular}} \vspace{2.5cm} \\
%\begin{tabular}[c]{@{}c@{}}\jurunoname \\ Jurado del Trabajo Final\end{tabular}     &  & \begin{tabular}[c]{@{}c@{}}\jurdosname\\ Jurado del Trabajo Final\end{tabular}  \vspace{2.5cm}  \\
%\multicolumn{3}{c}{\begin{tabular}[c]{@{}c@{}} \jurtresname\\ Jurado del Trabajo Final\end{tabular}} \vspace{.5cm}                                                                     
\end{tabular}
\end{table}


\section{Descripción técnica-conceptual del proyecto a realizar}
\label{sec:descripcion}

\begin{consigna}
EEl presente proyecto tiene como objetivo diseñar un sistema de monitoreo de estaciones de telefonía celular remotas. Actualmente no se dispone de este tipo de solución y se prevé su implementación gradual una vez testeado el prototipo en el interior de Colombia.
Se pretende disponer de este sistema de bajo costo en los primeros estadíos de la obra civil de construcción de la torre de la estación de modo de facilitar su seguimiento.
El sistema propuesto está pensado para poder realizar un seguimiento de eventos y monitoreo remoto del sitio, registrando:
\begin{itemize}
\item Fotografías de la torre periódicas y a solicitud que permitan visualizar el estado general
de las estructuras.
\item Registro de accesos, (fotografía), lector de RFID.
\item Registro de cortes de energía eléctrica.
\item Registro de condiciones ambientales como humedad y temperatura.
\item Envío de alerta ante intrusiones (a definir SMS/Email).
\item Opcional – Alarma sonora.
\item Opcional – Apertura de puerta por candado bluetooth.
\end{itemize}

En la Figura \ref{fig:diagBloques} se presenta el diagrama en bloques del sistema propuesto. 

\vspace{25px}

\begin{figure}[htpb]
\centering 
\includegraphics[width=\textwidth]{./Figuras/diagBloques.png}
\caption{Diagrama en bloques del sistema}
\label{fig:diagBloques}
\end{figure}

\vspace{25px}

\end{consigna}


\section{Identificación y análisis de los interesados}
\label{sec:interesados}

\begin{consigna} 
\begin{table}[ht]
%\caption{Identificación de los interesados}
%\label{tab:interesados}
\begin{tabularx}{\linewidth}{@{}|l|X|X|l|@{}}
\hline
\rowcolor[HTML]{C0C0C0} 
Rol           & Nombre y Apellido & Organización 	& Puesto 	\\ \hline
Auspiciante   & \clientename      & \empclientename	& COO      	\\ \hline
Cliente       & \clientename      & \empclientename	& COO      	\\ \hline
Impulsor      & Alejandro Ochoa   & \empclientename	& CEO      	\\ \hline
Responsable   & \authorname       & FIUBA        	& Alumno 	\\ \hline
% Colaboradores & -                 & -             	& -       	\\ \hline
Orientador    & \supname	      & \pertesupname 	& Director	Trabajo final \\ \hline
Equipo        & Matias Ventura | Paula Zanetti     & -  & - 	\\ \hline
% Opositores    & -                  & -             	& -       	\\ \hline
Usuario final externo & Equipos de Mantenimiento de Operadores  & Varios    & Varios- \\ \hline
Usuario final interno & Equipo de Operaciones      & \empclientename & Varios    	\\ \hline
\end{tabularx}
\end{table}

Características de stakeholders:
\begin{itemize}
\item Auspiciante: suele demorar en responder ante solicitud de provisión de detalles. 
\item Impulsor: es exigente en cuanto a plazos de entrega.
\item Equipo: Paula Zanetti puede colaborar en documentación UML aunque tiene tiempo escaso en su agenda.
\item Equipo: Matias Ventura puede proporcionar ayuda en desarrollo web y aspectos técnicos.
%\item Orientador: María Gómez, nos va a poder ayudar mucho con la gestión de impuestos.
\end{itemize}

\end{consigna}


\section{1. Propósito del proyecto}
\label{sec:proposito}

\begin{consigna}
EEl propósito de este proyecto es desarrollar un sistema IoT deberá permitir realizar un seguimiento y controles de la obra civil, eléctrica y metalmecánica, mediante toma de fotografías y registro de eventos en sitios remotos donde se emplaza una estación de telefonía celular.
\end{consigna}

\section{2. Alcance del proyecto}
\label{sec:alcance}

\begin{consigna}
EEl presente proyecto incluye el desarrollo de los siguientes componentes:
\item a) Hardware: Un prototipo basado en raspberry Pi4, el cual deberá almacenar la información obtenida por sensores y cámara en caso de no tener conexión a Internet. Al momento de  conectarse a Internet, deberá hacer una carga de los datos a la infraestructura Cloud.
\item b) Software: Un desarrollo en 3 capas residente en infraestructura Cloud a definir (AWS/Azure/GCP) que consta de una aplicación de backend que procese los datos obtenidos y los almacene en una base de datos.
Los usuarios accederán por una aplicación front-end a efectos de poder visualizar la información en un dashboard. 
El presente proyecto no incluye la fabricación en serie del dispositivo final para ser emplazado en los sitios.

\end{consigna}


\section{3. Supuestos del proyecto}
\label{sec:supuestos}

\begin{consigna}
PPara el desarrollo del presente proyecto se asumen los siguientes supuestos como verdaderos: 

\begin{enumerate}
\item Se dispone del hardware necesario para hacer el prototipo.
\item Se dispone del dinero necesario para realizar las adquisiciones necesarias.
\item El cliente provee suficiente detalle para poder realizar el desarrollo acorde a sus necesidades.
\item Se dispone de acceso a infraestructura Cloud para hosting del software y datos.
\item No hay restricciones para instalar este tipo de hardware en los sitios remotos.
\item No habrá cuestiones de fuerza mayor que impidan desarrollar el presente proyecto.
\end{enumerate}

\end{consigna}

\section{4. Requerimientos}
\label{sec:requerimientos}

\begin{consigna}
\begin{enumerate}
\item Grupo de requerimientos asociados con el dispositivo de hardware:
	\begin{enumerate}
	\item Alimentación eléctrica a 220 V / 50 Hz / conector tipo C / I para el prototipo de test.	
	\item Carcaza con grado de proteccion IP 65 - resistente a humedad y polvo- dado que estará en intemperie.
	\item Conector a paquete de baterías con una autonomía de al menos 2 horas.
	\end{enumerate}
\item Grupo de requerimientos asociados con software
	\begin{enumerate}
	\item Acceso a la información mediante un portal web disponible en Internet.
	\item Acceso a la aplicación basada en roles (Admin, usuario) con autenticación por API con Azure o similar. 
	\item Acceso a la información de cada sitio asignable por perfiles de usuario.
	\item Envío de alertas por SMS y/o email a elección del usuario.
	\item Almacenamiento de datos históricos por 5 años.
	\end{enumerate}
\end{enumerate}

\end{consigna}

\section{Historias de usuarios (\textit{Product backlog})}
\label{sec:backlog}

\begin{consigna} 
\begin{table}
\begin{tabularx}{\linewidth}{@{}|l|X|X|l|l|@{}}
\hline
\rowcolor[HTML]{C0C0C0} 
ID  & Hist. de Usuario & Crit. Aceptación & Prior. & Pond. \\ \hline
HU1 & COMO Jefe de Operaciones 
QUIERO Visualizar Estado de las estaciones 
PARA Detectar anomalias      & 	 & 1   & 5 	\\ \hline
HU2 & COMO Jefe de Operaciones 
QUIERO Recibir Alertas por SMS/EMail 
PARA Enviar un Ing. de Campo/FFSS de ser preciso   & 	& 2     & 5 	\\ \hline
HU3 & COMO Gerente de Operaciones 
QUIERO Visualizar fotos de avances de Obra 
PARA Validar pagos de avance de Obra  & 	&  4  & 8   	\\ \hline
HU4 & COMO CEO 
QUIERO Ver el estado general de las estaciones  
PARA Poder informar a inversores       & &  2 & 5	\\ \hline
HU5 & COMO COO 
QUIERO Ver listado de ingresos a las estaciones 
PARA Controlar actividades de Ing. de Campo &           	&   1  & 5  	\\ \hline
HU6 & COMO Jefe de Operaciones 
QUIERO Agregar Estaciones al sistema 
PARA Monitoreo de eventos  &  	& 1  & 3 \\ \hline
HU7 & COMO Jefe de Operaciones 
QUIERO Poder suspender envío de alertas 
PARA Permitir ventana de Mantenimiento  &   &  1 &  3	\\ \hline
HU8 & COMO Jefe de Operaciones 
QUIERO Poder crear usuarios en el sistema 
PARA Permitir acceso a información  &   &  3 &  3	\\ \hline

\end{tabularx}
\end{table}

Criterio Prioridad: de 1 a 5, siendo 1 más prioritario. 
Criterio Complejidad: de 3 a 13, siendo 3 menos complejo. 

\end{consigna}

\section{5. Entregables principales del proyecto}
\label{sec:entregables}

\begin{consigna}
LLista de entregables preliminar: 
\begin{itemize}
\item Prototipo hardware IoT.
\item Manual de uso perfil usuario estándar y usuario administrador.
\item Sitio web con los datos del prototipo.
\item Diagrama esquemático de la parte hardware para ser fabricado en serie.
\item Código fuente de la aplicación.
\item Diagrama de instalación.
\item Informe final.

\end{itemize}

\end{consigna}

\section{6. Desglose del trabajo en tareas}
\label{sec:wbs}

\begin{consigna}

\begin{enumerate}

\item Planificación

        \begin{enumerate}
        \item Definición de alcance (8 hs)
        \item Identificación de stakeholders (12 hs)
        \item Documentación -actualizar el presente documento- (8 hs)
        \end{enumerate}

\item Análisis

        \begin{enumerate}
        \item Requerimientos funcionales y no funcionales (24 hs)
        \item Identificación de actores primarios y secundarios (8 hs)
        \item Casos de Uso/Hist. usuario (24 hs)
        \item Diagramas UML (24 hs)
        \end{enumerate}

\item Capacitación

        \begin{enumerate}
        \item Aprendizaje Cloud Services (40 hs)
        \item Aprendizaje Desarrollo web  (40 hs)
        \item Aprendizaje Base de datos  (20 hs)
        \item Aprendizaje Redes  (16 hs)
        \end{enumerate}

\item Arquitectura

        \begin{enumerate}
        \item Investigación de arquitecturas (16 hs)
        \item Evaluación de alternativas (8 hs)
        \item Documentación de alternativa seleccionada (4 hs)
        \end{enumerate}

\item Diseño

        \begin{enumerate}
        \item Diseño hardware IoT (30 hs)
        \item Carcaza  (12 hs)
        \item Diseño Software (24 hs)
        
\item Codificación / Ejecución

        \begin{enumerate}
        \item Programación Front-end (30 hs)
        \item Programación Backend (30 hs)
        \item Programación Base de datos (16 hs)
        \item Configuración infraestructura de cloud (16 hs)
        \item Validación de seguridad OWASP (16 hs)
        \item Compra Hardware y accesorios (8 hs)
        \item Compra certificado digital, dominio, Cloud services (16 hs)
        \end{enumerate}

\item Integración

        \begin{enumerate}
        \item Conexión de componentes (16 hs)
        \item Troubleshooting (8 hs)
        \end{enumerate}

\item Validación

        \begin{enumerate}
        \item Testing alfa (12 hs)
        \item Testing beta (12 hs)
        \item Stress Testing (8 hs)
        \end{enumerate}

\item Entrega

        \begin{enumerate}
        \item Manuales de usuarios (12 hs)
        \item Capacitación de usuarios (20 hs)
        \item Seguimiento inicial (12 hs)
        \end{enumerate}

\item Gestión de Proyecto

        \begin{enumerate}
        \item Documentación de proyecto (20 hs)
        \item Seguimiento con director de proyecto (20 hs)
        \item Cierre de proyecto (20 hs)
        \end{enumerate}

\end{enumerate}

Cantidad total de horas: (612 hs)

\end{consigna}

\section{7. Diagrama de Activity On Node}
\label{sec:AoN}

\begin{consigna}{red}
Armar el AoN a partir del WBS definido en la etapa anterior. 

%La figura \ref{fig:AoN} fue elaborada con el paquete latex tikz y pueden consultar la siguiente referencia \textit{online}:

%\url{https://www.overleaf.com/learn/latex/LaTeX_Graphics_using_TikZ:_A_Tutorial_for_Beginners_(Part_3)\%E2\%80\%94Creating_Flowcharts}

\end{consigna}

\begin{figure}[htpb]
\centering 
\includegraphics[width=.8\textwidth]{./Figuras/AoN.png}
\caption{Diagrama en \textit{Activity on Node}}
\label{fig:AoN}
\end{figure}

Indicar claramente en qué unidades están expresados los tiempos.
De ser necesario indicar los caminos semicríticos y analizar sus tiempos mediante un cuadro.
Es recomendable usar colores y un cuadro indicativo describiendo qué representa cada color, como se muestra en el siguiente ejemplo:



\section{8. Diagrama de Gantt}
\label{sec:gantt}

\begin{consigna}{red}
Utilizar el software Gantter for Google Drive o alguno similar para dibujar el diagrama de Gantt.

Existen muchos programas y recursos \textit{online} para hacer diagramas de gantt, entre las cuales destacamos:

\begin{itemize}
\item Planner
\item GanttProject
\item Trello + \textit{plugins}. En el siguiente link hay un tutorial oficial: \\ \url{https://blog.trello.com/es/diagrama-de-gantt-de-un-proyecto}
\item Creately, herramienta online colaborativa. \\\url{https://creately.com/diagram/example/ieb3p3ml/LaTeX}
\item Se puede hacer en latex con el paquete \textit{pgfgantt}\\ \url{http://ctan.dcc.uchile.cl/graphics/pgf/contrib/pgfgantt/pgfgantt.pdf}
\end{itemize}

Pegar acá una captura de pantalla del diagrama de Gantt, cuidando que la letra sea suficientemente grande como para ser legible. 
Si el diagrama queda demasiado ancho, se puede pegar primero la ``tabla'' del Gantt y luego pegar la parte del diagrama de barras del diagrama de Gantt.

Configurar el software para que en la parte de la tabla muestre los códigos del EDT (WBS).\\
Configurar el software para que al lado de cada barra muestre el nombre de cada tarea.\\
Revisar que la fecha de finalización coincida con lo indicado en el Acta Constitutiva.

En la figura \ref{fig:gantt}, se muestra un ejemplo de diagrama de gantt realizado con el paquete de \textit{pgfgantt}. En la plantilla pueden ver el código que lo genera y usarlo de base para construir el propio.

\begin{figure}[htbp]
\begin{center}
\begin{ganttchart}{1}{12}
  \gantttitle{2020}{12} \\
  \gantttitlelist{1,...,12}{1} \\
  \ganttgroup{Group 1}{1}{7} \\
  \ganttbar{Task 1}{1}{2} \\
  \ganttlinkedbar{Task 2}{3}{7} \ganttnewline
  \ganttmilestone{Milestone o hito}{7} \ganttnewline
  \ganttbar{Final Task}{8}{12}
  \ganttlink{elem2}{elem3}
  \ganttlink{elem3}{elem4}
\end{ganttchart}
\end{center}
\caption{Diagrama de gantt de ejemplo}
\label{fig:gantt}
\end{figure}

\end{consigna}

\section{9. Matriz de uso de recursos de materiales}
\label{sec:recursos}


\begin{table}
\label{tab:recursos}
\centering
\begin{tabularx}{\linewidth}{@{}|c|X|X|X|X|c|@{}}
\hline
\cellcolor[HTML]{C0C0C0} & \cellcolor[HTML]{C0C0C0} & \multicolumn{4}{c|}{\cellcolor[HTML]{C0C0C0}Recursos requeridos (horas)} \\ \cline{3-6} 
\multirow{-2}{*}{\cellcolor[HTML]{C0C0C0}\begin{tabular}[c]{@{}c@{}}Código\\ WBS\end{tabular}} & \multirow{-2}{*}{\cellcolor[HTML]{C0C0C0}\begin{tabular}[c]{@{}c@{}}Nombre \\ tarea\end{tabular}} & Material 1 & Material 2 & Material 3 & Material 4 \\ \hline
 &  &  &  &  &  \\ \hline
 &  &  &  &  &  \\ \hline
 &  &  &  &  &  \\ \hline
 &  &  &  &  &  \\ \hline
 &  &  &  &  &  \\ \hline
 &  &  &  &  &  \\ \hline
 &  &  &  &  &  \\ \hline
 &  &  &  &  &  \\ \hline 
 &  &  &  &  &  \\ \hline
 &  &  &  &  &  \\ \hline
 &  &  &  &  &  \\ \hline
 &  &  &  &  &  \\ \hline
 &  &  &  &  &  \\ \hline
 &  &  &  &  &  \\ \hline
 &  &  &  &  &  \\ \hline
 &  &  &  &  &  \\ \hline
 &  &  &  &  &  \\ \hline
 &  &  &  &  &  \\ \hline
 &  &  &  &  &  \\ \hline
 &  &  &  &  &  \\ \hline
 &  &  &  &  &  \\ \hline
 &  &  &  &  &  \\ \hline
 &  &  &  &  &  \\ \hline
 &  &  &  &  &  \\ \hline 
 &  &  &  &  &  \\ \hline
 &  &  &  &  &  \\ \hline
 &  &  &  &  &  \\ \hline
 &  &  &  &  &  \\ \hline

\end{tabularx}%
\end{table}


\section{10. Presupuesto detallado del proyecto}
\label{sec:presupuesto}

\begin{consigna}{red}
Si el proyecto es complejo entonces separarlo en partes:
\begin{itemize}
\item Un total global, indicando el subtotal acumulado por cada una de las áreas.
\item El desglose detallado del subtotal de cada una de las áreas.
\end{itemize}

IMPORTANTE: No olvidarse de considerar los COSTOS INDIRECTOS.

\end{consigna}

\begin{table}[htpb]
\centering
\begin{tabularx}{\linewidth}{@{}|X|c|r|r|@{}}
\hline
\rowcolor[HTML]{C0C0C0} 
\multicolumn{4}{|c|}{\cellcolor[HTML]{C0C0C0}COSTOS DIRECTOS} \\ \hline
\rowcolor[HTML]{C0C0C0} 
Descripción &
  \multicolumn{1}{c|}{\cellcolor[HTML]{C0C0C0}Cantidad} &
  \multicolumn{1}{c|}{\cellcolor[HTML]{C0C0C0}Valor unitario} &
  \multicolumn{1}{c|}{\cellcolor[HTML]{C0C0C0}Valor total} \\ \hline
 &
  \multicolumn{1}{c|}{} &
  \multicolumn{1}{c|}{} &
  \multicolumn{1}{c|}{} \\ \hline
 &
  \multicolumn{1}{c|}{} &
  \multicolumn{1}{c|}{} &
  \multicolumn{1}{c|}{} \\ \hline
\multicolumn{1}{|l|}{} &
   &
   &
   \\ \hline
\multicolumn{1}{|l|}{} &
   &
   &
   \\ \hline
\multicolumn{3}{|c|}{SUBTOTAL} &
  \multicolumn{1}{c|}{} \\ \hline
\rowcolor[HTML]{C0C0C0} 
\multicolumn{4}{|c|}{\cellcolor[HTML]{C0C0C0}COSTOS INDIRECTOS} \\ \hline
\rowcolor[HTML]{C0C0C0} 
Descripción &
  \multicolumn{1}{c|}{\cellcolor[HTML]{C0C0C0}Cantidad} &
  \multicolumn{1}{c|}{\cellcolor[HTML]{C0C0C0}Valor unitario} &
  \multicolumn{1}{c|}{\cellcolor[HTML]{C0C0C0}Valor total} \\ \hline
\multicolumn{1}{|l|}{} &
   &
   &
   \\ \hline
\multicolumn{1}{|l|}{} &
   &
   &
   \\ \hline
\multicolumn{1}{|l|}{} &
   &
   &
   \\ \hline
\multicolumn{3}{|c|}{SUBTOTAL} &
  \multicolumn{1}{c|}{} \\ \hline
\rowcolor[HTML]{C0C0C0}
\multicolumn{3}{|c|}{TOTAL} &
   \\ \hline
\end{tabularx}%
\end{table}


\section{11. Matriz de asignación de responsabilidades}
\label{sec:responsabilidades}
\begin{consigna}{red}
Establecer la matriz de asignación de responsabilidades y el manejo de la autoridad completando la siguiente tabla:

\begin{table}[htpb]
\centering
\resizebox{\textwidth}{!}{%
\begin{tabular}{|c|c|c|c|c|c|}
\hline
\rowcolor[HTML]{C0C0C0} 
\cellcolor[HTML]{C0C0C0} &
  \cellcolor[HTML]{C0C0C0} &
  \multicolumn{4}{c|}{\cellcolor[HTML]{C0C0C0}Listar todos los nombres y roles del proyecto} \\ \cline{3-6} 
\rowcolor[HTML]{C0C0C0} 
\cellcolor[HTML]{C0C0C0} &
  \cellcolor[HTML]{C0C0C0} &
  Responsable &
  Orientador &
  Equipo &
  Cliente \\ \cline{3-6} 
\rowcolor[HTML]{C0C0C0} 
\multirow{-3}{*}{\cellcolor[HTML]{C0C0C0}\begin{tabular}[c]{@{}c@{}}Código\\ WBS\end{tabular}} &
  \multirow{-3}{*}{\cellcolor[HTML]{C0C0C0}Nombre de la tarea} &
  \authorname &
  \supname &
  Nombre de alguien &
  \clientename \\ \hline
 &  &  &  &  &  \\ \hline
 &  &  &  &  &  \\ \hline
 &  &  &  &  &  \\ \hline
\end{tabular}%
}
\end{table}

{\footnotesize
Referencias:
\begin{itemize}
	\item P = Responsabilidad Primaria
	\item S = Responsabilidad Secundaria
	\item A = Aprobación
	\item I = Informado
	\item C = Consultado
\end{itemize}
} %footnotesize

Una de las columnas debe ser para el Director, ya que se supone que participará en el proyecto.
A su vez se debe cuidar que no queden muchas tareas seguidas sin ``A'' o ``I''.

Importante: es redundante poner ``I/A'' o ``I/C'', porque para aprobarlo o responder consultas primero la persona debe ser informada.

\end{consigna}

\section{12. Gestión de riesgos}
\label{sec:riesgos}

\begin{consigna}{red}
a) Identificación de los riesgos (al menos cinco) y estimación de sus consecuencias:
 
Riesgo 1: detallar el riesgo (riesgo es algo que si ocurre altera los planes previstos)
\begin{itemize}
\item Severidad (S): mientras más severo, más alto es el número (usar números del 1 al 10).\\
Justificar el motivo por el cual se asigna determinado número de severidad (S).
\item Probabilidad de ocurrencia (O): mientras más probable, más alto es el número (usar del 1 al 10).\\
Justificar el motivo por el cual se asigna determinado número de (O). 
\end{itemize}   

Riesgo 2:
\begin{itemize}
\item Severidad (S): 
\item Ocurrencia (O):
\end{itemize}

Riesgo 3:
\begin{itemize}
\item Severidad (S): 
\item Ocurrencia (O):
\end{itemize}


b) Tabla de gestión de riesgos:      (El RPN se calcula como RPN=SxO)

\begin{table}[htpb]
\centering
\begin{tabularx}{\linewidth}{@{}|X|c|c|c|c|c|c|@{}}
\hline
\rowcolor[HTML]{C0C0C0} 
Riesgo & S & O & RPN & S* & O* & RPN* \\ \hline
       &   &   &     &    &    &      \\ \hline
       &   &   &     &    &    &      \\ \hline
       &   &   &     &    &    &      \\ \hline
       &   &   &     &    &    &      \\ \hline
       &   &   &     &    &    &      \\ \hline
\end{tabularx}%
\end{table}

Criterio adoptado: 
Se tomarán medidas de mitigación en los riesgos cuyos números de RPN sean mayores a...

Nota: los valores marcados con (*) en la tabla corresponden luego de haber aplicado la mitigación.

c) Plan de mitigación de los riesgos que originalmente excedían el RPN máximo establecido:
 
Riesgo 1: plan de mitigación (si por el RPN fuera necesario elaborar un plan de mitigación).
  Nueva asignación de S y O, con su respectiva justificación:
  - Severidad (S): mientras más severo, más alto es el número (usar números del 1 al 10).
          Justificar el motivo por el cual se asigna determinado número de severidad (S).
  - Probabilidad de ocurrencia (O): mientras más probable, más alto es el número (usar del 1 al 10).
          Justificar el motivo por el cual se asigna determinado número de (O).

Riesgo 2: plan de mitigación (si por el RPN fuera necesario elaborar un plan de mitigación).
 
Riesgo 3: plan de mitigación (si por el RPN fuera necesario elaborar un plan de mitigación).

\end{consigna}


\section{13. Gestión de la calidad}
\label{sec:calidad}

\begin{consigna}{red}
Para cada uno de los requerimientos del proyecto indique:
\begin{itemize} 
\item Req \#1: copiar acá el requerimiento.

Verificación y validación:

\begin{itemize}
\item Verificación para confirmar si se cumplió con lo requerido antes de mostrar el sistema al cliente. Detallar 
\item Validación con el cliente para confirmar que está de acuerdo en que se cumplió con lo requerido. Detallar  
\end{itemize}

\end{itemize}

Tener en cuenta que en este contexto se pueden mencionar simulaciones, cálculos, revisión de hojas de datos, consulta con expertos, mediciones, etc.

\end{consigna}

\section{14. Comunicación del proyecto}
\label{sec:comunicaciones}

El plan de comunicación del proyecto es el siguiente:

\begin{table}[htpb]
\centering
\begin{tabularx}{\linewidth}{@{}|X|C{2.4cm}|C{3cm}|C{1.8cm}|C{2cm}|C{2.1cm}|@{}}
\hline
\rowcolor[HTML]{C0C0C0} 
\multicolumn{6}{|c|}{\cellcolor[HTML]{C0C0C0}PLAN DE COMUNICACIÓN DEL PROYECTO}           \\ \hline
\rowcolor[HTML]{C0C0C0} 
¿Qué comunicar? & Audiencia & Propósito & Frecuencia & Método de comunicac. & Responsable \\ \hline
                &           &           &            &                      &             \\ \hline
                &           &           &            &                      &             \\ \hline
                &           &           &            &                      &             \\ \hline
                &           &           &            &                      &             \\ \hline
                &           &           &            &                      &             \\ \hline
\end{tabularx}
\end{table}

\section{15. Gestión de compras}
\label{sec:compras}

\begin{consigna}{red}
En caso de tener que comprar elementos o contratar servicios:
a) Explique con qué criterios elegiría a un proveedor.
b) Redacte el Statement of Work correspondiente.
\end{consigna}

\section{16. Seguimiento y control}
\label{sec:seguimiento}

\begin{consigna}{red}
Para cada tarea del proyecto establecer la frecuencia y los indicadores con los se seguirá su avance y quién será el responsable de hacer dicho seguimiento y a quién debe comunicarse la situación (en concordancia con el Plan de Comunicación del proyecto).

El indicador de avance tiene que ser algo medible, mejor incluso si se puede medir en \% de avance. Por ejemplo,se pueden indicar en esta columna cosas como ``cantidad de conexiones ruteadeas'' o ``cantidad de funciones implementadas'', pero no algo genérico y ambiguo como ``\%'', porque el lector no sabe porcentaje de qué cosa.

\end{consigna}

\begin{longtable}{|m{1cm}|m{3.5cm}|m{2.2cm}|m{2cm}|m{3cm}|m{1.5cm}|}
\hline
\rowcolor[HTML]{C0C0C0} 
\multicolumn{6}{|c|}{\cellcolor[HTML]{C0C0C0}SEGUIMIENTO DE AVANCE}                                                                       \\ \hline
\rowcolor[HTML]{C0C0C0} 
Tarea del WBS 			& Indicador de avance & Frecuencia de reporte & Resp. de seguimiento & Persona a ser informada & Método de comunic. \\ \hline
\endfirsthead

\hline
\rowcolor[HTML]{C0C0C0} 
\multicolumn{6}{c}{\cellcolor[HTML]{C0C0C0}SEGUIMIENTO DE AVANCE}                                                                       \\ \hline
\rowcolor[HTML]{C0C0C0} 
Tarea del WBS 			& Indicador de avance & Frecuencia de reporte & Resp. de seguimiento & Persona a ser informada & Método de comunic. \\ \hline
\endhead

\multicolumn{6}{c}{Continúa}
\endfoot

\endlastfoot

1.1	& Fecha de inicio  & Única vez al comienzo & \authorname & \clientename, \supname & email \\ \hline
2.1	& Avance de las subtareas  & Mensual mientras dure la tarea & \authorname & \clientename, \supname & email \\ \hline

\end{longtable}

\begin{table}[!htpb]
\centering
%\begin{tabularx}{\linewidth}{@{}|X|X|X|X|X|X|@{}}
\begin{tabularx}{\linewidth}{@{}|X|C{2.5cm}|C{3cm}|C{2cm}|C{2cm}|C{2.5cm}|@{}}
\hline
\rowcolor[HTML]{C0C0C0} 
\multicolumn{6}{|c|}{\cellcolor[HTML]{C0C0C0}SEGUIMIENTO DE AVANCE}                                                                       \\ \hline
\rowcolor[HTML]{C0C0C0} 
Tarea del WBS & Indicador de avance & Frecuencia de reporte & Resp. de seguimiento & Persona a ser informada & Método de comunic. \\ \hline
 &  &  &  &  &  \\ \hline
 &  &  &  &  &  \\ \hline
 &  &  &  &  &  \\ \hline
 &  &  &  &  &  \\ \hline
 &  &  &  &  &  \\ \hline
\end{tabularx}%
%}
\end{table}

\section{17. Procesos de cierre}    
\label{sec:cierre}

\begin{consigna}{red}
Establecer las pautas de trabajo para realizar una reunión final de evaluación del proyecto, tal que contemple las siguientes actividades:

\begin{itemize}
\item Pautas de trabajo que se seguirán para analizar si se respetó el Plan de Proyecto original:
 - Indicar quién se ocupará de hacer esto y cuál será el procedimiento a aplicar. 
\item Identificación de las técnicas y procedimientos útiles e inútiles que se utilizaron, y los problemas que surgieron y cómo se solucionaron:
 - Indicar quién se ocupará de hacer esto y cuál será el procedimiento para dejar registro.
\item Indicar quién organizará el acto de agradecimiento a todos los interesados, y en especial al equipo de trabajo y colaboradores:
  - Indicar esto y quién financiará los gastos correspondientes.
\end{itemize}

\end{consigna}


\end{document}
